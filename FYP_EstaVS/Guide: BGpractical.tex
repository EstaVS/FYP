%======================================================================================== 
%Preamble. Latex needs all this stuff to work, but it doesn't appear in your document.

\documentclass{article}[12pt]
\usepackage{geometry}                
\usepackage{graphicx}
\usepackage{amsmath}
\usepackage{amssymb}
\usepackage{float}
\floatstyle{plain}
\usepackage[]{microtype}
\usepackage{listings}
\lstset{
    literate={~} {$\sim$}{1}
}
\usepackage{color}
\usepackage{epstopdf}

\usepackage{lipsum}
\newcommand{\eqname}[1]{\tag*{\bf #1}}% Tag equation with name
%\DeclareGraphicsRule{.tif}{png}{.png}{`convert #1 `dirname #1`/`basename #1 .tif`.png}
\usepackage{hyperref}
\hypersetup{
    colorlinks=true,
    linkcolor=magenta,
    filecolor=magenta,      
    urlcolor=blue,
    citecolor={blue}
    }
    
\usepackage[backend=biber, style=authoryear, uniquename=false,doi=false,isbn=false, url=false, giveninits]{biblatex}
\addbibresource{Wormreferences.bib}



%\usepackage{biblatex}
%==========================================================
% My title, name, and date
\title{\emph{lin-22}: a Driving Force in the Early Development of \textit{Caenorhabditis elegans}}
\author{Emma Guimaraes}
\date{\today}                                         
%==========================================================


\begin{document}

\maketitle
\tableofcontents
\newpage

%==============This first section is just teaching you how to use LaTeX. It's not part of your paper.



\section{Using LaTeX}

\subsection{Using LaTex in Overleaf}

It's a pain, but you'll get used to it.  Google to find stuff out, ask each other, \emph{in extremis}, ask me. 

Here I am going to write something else. And how do I italicize something?  In Word you'd use a drop down menu or go command+I or something. But that's too easy for scientists! We like doing stuff the hard way.  This is \emph{how}! And \textbf{bold}.  And ``quote''And you cite \cite{Levin2012}. 
\subsection{Citation}
This is how you cite \cite{Levin2012} or \parencite{Goldstein1995}.

\subsection{Figures, Tables \& Equations}

You can put figures directly into the text. If you want to mess with their formatting, look \href{http://en.wikibooks.org/wiki/LaTeX/Floats,_Figures_and_Captions}{here}.

Figure \ref{fig:rpl} is really great. 

\floatstyle{plain}
\restylefloat{figure}
\begin{figure}[h]
\centering
\includegraphics[width=7cm]{rpl}
\caption{\emph{rpl} gene expression in \emph{C. elegans} development}
\label{fig:rpl}
\end{figure}

Figure \ref{fig:rpl} is a bit of a rubbish figure. You should generate better ones in R by playing with the plot settings --- or use the ggplot2 library. It's very nice, but has quite a steep learning curve. \LaTeX is pretty good about placing figures, but of course if it doesn't have the space on a page, it will put it somewhere that you may not want it. You'll just have to experiment. 


Here is how to make a table such as Table \ref{table:things table}:
\begin{table}[ht]
\begin{center}
\begin{tabular}{lrrrrr}%{| p{1cm} | p{1cm} | p{1cm} |p{1cm} |p{1cm} |p{1cm}|}
\hline
Things&$r$&$r'$&$n$&$n'$&$w_i$\\
\hline
A&10&1&1&10&10 \\
B&10&6&1&5&5\\
C&10&3&1&8&8\\
D&10&2&1&9&9\\
E&10&5&1&6&6\\
\hline
\\
\end{tabular}
\caption{The Table's title}
\label{table:things table}
\end{center}
\end{table}

And, finally, here is how to make an equation such as Equation \ref{eqn:nonlin}. These can be as complicated as you please. 

\begin{equation}
\Delta SD(z)= Var(z_i-\bar z)^2\beta\left(\frac{w_i}{\bar w}, (z_i-\bar z)^2\right)
\label{eqn:nonlin}
\end{equation}



\clearpage

%Your actual paper begins here.=======================
% You need not use this format, but it's not a bad one. 
\section{Introduction}
%some random text

\subsection{A subsection}

\subsubsection{A subsubsection}

\section{Methods}

%Write about the methods you used.  You don't have to give R code, just tell me what packages you used, and if you did anything special tell me what. Nor do you have to use these subsections---they're just examples. 




\section{Results}
\subsection{\emph{rpl-39} shows dynamic expression levels during early development}


\subsection{Many of \emph{rpl-39}'s interactors cluster into three groups}



\section{Discussion}


\section{Conclusions}

\printbibliography


\end{document}